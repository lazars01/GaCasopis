% !TEX encoding = UTF-8 Unicode
\documentclass[hyperref={bookmarks=false}]{beamer}

\usetheme{CambridgeUS}
\setbeamercolor{bibliography entry author}{fg=black}
\setbeamercolor{item}{fg=darkred}
\setbeamercolor{caption name}{fg=darkred}
\beamertemplatenavigationsymbolsempty
\bibliography{references}



\usepackage[serbianc]{babel}
\usepackage{type1ec}
\usepackage{cmap}
\usepackage{tikz}
\usepackage{url}
\usepackage{amsmath} 
\usepackage{biblatex}

\title{Aproksimacija najdužeg razapinjućeg stabla sa okolinama}
\author[Lazar Stanojević]{Lazar Stanojević\\\small mi231013@alas.matf.bg.ac.rs}
\institute[]{Geometrijski algoritmi\\Matematički fakultet, Univerzitet u Beogradu}
\date[Matematički fakultet]{28. novembar 2023.}

\begin{document}

\frame{\titlepage}

\begin{frame}{Sadržaj}
\tableofcontents[subsectionstyle=show]
\end{frame}

\section{Opšte o autorima i radu}
\subsection{Autori}
\begin{frame}{Autori}
\begin{itemize}
\item Journal of Computational Geometry, Vol 14. No. 1 (2023)
\item Tema: Approximating longest spanning tree with neighborhoods
\item \textbf{Ahmad Biniaz}
\begin{itemize}
    \item \url{https://cglab.ca/~biniaz/} - lična stranica
    \item ahmad.biniaz@gmail.com
    \item PhD in Computer Science from Carleton University, Ottawa, Canada, 2013-2017.
    \item Interesovanja: algoritmi i strukture podataka, diskretna i računarska geometrija, diskretna matematika
\end{itemize}
\end{itemize}
\end{frame}

\subsection{Opis problema}
\begin{frame}{Opis problema}
\begin{itemize}
    \item Problem maksimizacije u Euklidskoj ravni
    \item Dat je skup okolina/susedstva (unija prostih poligona), ne nužno disjunktnih
    \item Potrebno izabrati po tačku iz svake okoline, tako da najduže razapinjuće stablo nad tim tačkama ima maksimalnu dužinu
\end{itemize}
\end{frame}


\section{Osnovni pojmovi}
\begin{frame}{Osnovni pojmovi}
\begin{itemize}
\item \textit{Zvezda} sa centrom u čvoru p, je stablo u kom je svaka grana susedna sa čvorom p
\item \textit{Dupla zvezda} sa centrom u čvorovima p i q, je stablo koje sadrži granu pq i svaka druga grana je susedna ili sa p ili sa q
\item \textit{Najmanji zatvarajući disk} skupa tačaka P, je najmanji disk koji sadrži sve tačke iz P
\item \textit{Dijametralni par} skupa P su tačke p i q takve da dostižu najveće euklidsko rastojanje
\item \textit{Bihromatski dijametralni par} skupa P čine tačke p i q, različite boje, takve da je rastojanje između njih maksimalno, uz pretpostavku da su tačke u skupu P obojene
\item \textit{Centar mase} skupa P (centroid), je tačka \textit{m} u ravni, takva da za svaku proizvolju tačku \textit{u} važi: 
\[
\sum_{p \in P} \overrightarrow{up} = |P| \cdot \overrightarrow{um}
\]
\end{itemize}
\end{frame}

\section{Jednostavan algoritam}
\begin{frame}{Jednostavan algoritam}
\begin{itemize}
\item Chen i Dumitrescu
\item Uzmimo bihromatski par (a, b) od tačaka iz n datih okolina
\item Formiramo zvezdu $S_a$, povezujući a sa b, i a sa proizvoljnom tačkom iz svake ostale okoline
\item Formiramo analogno $S_b$
\item Dužina svake grane optimalnog rešenja T je najviše |ab|, odatle je $len(T) \leq (n-1)\cdot|ab|$
\item Iz nejednakosti trougla imamo: $S_a + S_b \ge n\cdot|ab| \ge len(T)$
\item Sledi da je duža od dve zvezde 0.5-aproksimativno rešenje problema
\end{itemize}    
\end{frame}

\section{Napredniji algoritam}
\subsection{Opis}
\begin{frame}{Opis}
\begin{itemize}
    \item 0.548-aproksimativno rešenje problema
    \item Izračunavamo razapinjuću duplu zvezdu D, i najviše tri razapnjuće zvezde, $S_1, S_2$ i $S_3$
    \item Nađemo (a, b) bihromatski dijametralni par, i za svaku od preostalih okolina, nađemo tačku $p$ koja je najdalja od $a$, i tačku $q$ koja je najdalja od $b$
    \item Ako je $|ap| \ge |bq|$ u $D$ dodajemo $p$, inače dodajemo $q$
    \item Zbog nejednakosti trouglova, svaka grana u $D$ je dužine bar $|ab|/2$
    \item Neka je $C$ najmanji zatvarajući disk za $X$ (skup svih okolina)
    \item Ako $C$ sadrži tačno dve tačke iz $X$, onda zvezde centriramo u njima
    \item U suprotnom, biramo tri tačke tako da trougao koji formiraju sadrži centar diska $C$, i njih postavljamo za centre razapinjućih zvezdi
\end{itemize}
\end{frame}

\subsection{Složenost}
\begin{frame}{Složenost}
\begin{itemize}
    \item Najmanji zavarajući disk možemo da izračunamo u O(N) vremenskoj složenosti, gde je N broj tačaka koje opisuju svih $n$ okolina
    \item Bihromatski dijametralni par možemo naći za O($N\cdot \log N\cdot \log n$)
    \item Ostatak algoritma uzima linearno vreme, O(N)
\end{itemize}
\end{frame}

\subsection{Analiza faktora aproksimacije}
\begin{frame}{Analiza}
\begin{itemize}
    \item Glavna ideja: pokazati da ako je radijus diska $C$ bar $\delta = \frac{\sqrt{7}-1}{3} \approx 0.548$, onda je neka od zvezda $S_i$ željeno stablo, u suprotnom je to dupla zvezda $D$
    \item Lema 1: Ako je $r \ge \delta$ i granica diska $C$ sadrži tačno dve tačke, onda je $max(len(S_1), len(S_2)) \ge \delta \cdot len(T)$, gde je T optimalno rešenje
    \item Lema 2: Ako granica diska C sadrži tri ili više tački, onda za svaku tačku $m$ u ravni, postoji tačka $c_j \in {c_1, c_1, c_3}$ takva da je $|c_jm| \ge r$
    \item Lema 3: Ako je $r \ge \delta$ i granica diska $C$ sadrži tri ili više tačaka, onda je $max(len(S_1), len(S_2), len(S_3)) \ge \delta \cdot len(T)$
    \item Lema 4: Ako je $r \leq \delta$ onda je $len(D) \ge \delta \cdot len(T)$
\end{itemize}
\end{frame}

\section{Doprinos, primene i srodni problemi}
\subsection{Doprinos}
\begin{frame}{Doprinos}
\begin{itemize}
    \item Glavni doprinos rada je sledeća teorema:
    \begin{itemize}
        \item 0.548-aproksimacija može da bude izračunata u linearnom vremenu nakon nalaženja bihromatskog dijametra
    \end{itemize}
    \item Poboljšan je faktor aproksimacije, sa do tada najboljeg 0.517 na 0.548
    \item Dokazano je da algoritam koji uvek uključuje bihromatski dijametar u rešenje ne može da ima faktor aproksimacije veći od 0.5
\end{itemize}
\end{frame}
\subsection{Primene i srodni problemi}
\begin{frame}{Primene i srodni problemi}
\begin{itemize}
    \item Primene:
    \begin{itemize}
        \item Analiza najgoreg slučaja u raznim heuristikama, u kombinatornoj optimizaciji
        \item Aproksimacija maksimalnih triangulacija
        \item Algoritmi klasterovanja
    \end{itemize}
    \item Srodni problemi:
    \begin{itemize}
        \item Problem euklidskog Steinerovog stabla
        \item Problem trgovačkog putnika sa okolinama
        \item Konveksni omotač nepreciznih tačaka
    \end{itemize}
 
\end{itemize}   
\end{frame}

\section{Zaključak}
\begin{frame}{Zaključak}
\begin{itemize}
    \item Otvoren je problem daljeg poboljšanja algoritama aproksimacije
    \item Za to je potrebna detaljnija analiza
    \item Razmatrane su zvezde i dupla zvezda razdvojeno, možda bi posmatranje zajedno dovelo do poboljšanja
\end{itemize}
    
\end{frame}

\section{Literatura}
\begin{frame}{Literatura}
\begin{itemize}
    \item Approximating longest spanning tree with neighborhoods, Ahmad Biniaz, Journal of Computational Geometry, Vol. 14 No. 1, Pages 1-13, 4.4.2023. 
\end{itemize}
\end{frame}
\end{document}